\section{Introduction} %%%%%%%%%%%%%%%%%%%%%%%%%%%%%%%%%%%%%%%%%%%%%%%%%%%%%%%%%
\label{sec:intro}

%Conventional object-oriented programming practice suggests \emph{Visitor Design 
%Pattern}~\cite{DesignPatterns} for principled traversal of structured data such 
%as those manipulated by compilers, GUI frameworks, etc. However, our experience 
%with various \Cpp{} front-ends and program analysis frameworks~\cite{Pivot09,Phoenix,Clang} 
%have been rather unsatisfactory. We found visitors unsuitable to express our 
%application logic, surprisingly hard to teach students, and slow. We found 
%dynamic casts in many places, often nested. This suggested that users preferred 
%shorter, cleaner, and more direct codes to visitors, even at the high cost in 
%performance. Most of the dynamic cast usage mimicked the moral equivalent of 
%pattern matching as found in modern functional programming languages.

Pattern matching is an abstraction mechanism popularized by the functional
programming community, for example ML~\cite{ML78},
OCaml~\cite{OPM01}, and Haskell~\cite{haskell90},
and recently adopted by several multi-paradigm and object-oriented programming 
languages such as Scala~\cite{Scala2nd}, F\#~\cite{Syme07}, 
%Newspeak~\cite{geller2010pattern}, Fortress~\cite{RPS10}, 
%Thorn~\cite{Thorn2012}, Grace~\cite{Grace2012},
 and dialects of 
\Cpp{}\cite{Prop96,App}.
The expressive power of pattern matching has been cited as the number one reason for choosing a functional language for a 
task~\cite{FramaC09}, \cite{Minsky08}, \cite{Nanavati08}.
%Expressivity of pattern matching has been considered so crucial in certain application domains, 
%that it became the number one reason for choosing a functional language for the 
%job in those domains. In their experience report on implementing Frama-C static 
%analyzer, Cuoq et al~\cite{FramaC09} list expressiveness as the number one reason for using 
%OCaml instead of \Cpp{}. Similar sentiments are expressed in 
%two other experience reports detailing implementation of a high-performance 
%trading platform in OCaml~\cite{Minsky08} and implementation of a hardware-design 
%toolset in Haskell~\cite{Nanavati08}.

This paper presents a functional-style pattern matching extension to \Cpp{}.
%% Introduction of pattern matching into a new language is nontrivial;
%% introducing it into an existing language in widespread use is an even greater 
%% challenge. The obvious utility of the feature is easily compromised by the need 
%% to fit into the language's syntax, semantics, and toolchains. A prototype 
%% implementation requires more effort than for an experimental language.
%% It is also
%% harder to get into use because mainstream users are unwilling to try 
%% non-portable, non-standard, and unoptimized features, so we need to provide an optimized implementation. 
To allow experimentation and to be able to use production-quality tool chains (in particular, compilers and optimizers), we  implemented our matching facilities as a C++ library.
%To balance utility and effort we took the Semantically Enhanced Library Language 
%(SELL) approach~\cite{SELL,BSGDR05}, extending a general-purpose 
%programming language with a library.
%%  With pattern matching, we 
%% (somewhat surprisingly) managed to avoid external tool support by relying on 
%% some pretty nasty macro hacking and template meta-programming to provide a 
%% conventional and convenient interface to an efficient library implementation.

%Before we go 
%any further, though, we need to explicitly clarify one aspect. Throughout the 
%paper, when not stated otherwise, by pattern matching we mean run-time pattern 
%matching -- a facility that can be used to analyze and decompose run-time values.
%This is important, because \Cpp{} has a pure functional sublanguage in it with 
%striking similarity to ML and Haskell that has a very extensive compile-time 
%pattern-matching facility in it~\cite{Milewski11}. The sublanguage in question 
%is the template facilities of \Cpp{}, which allows one to encode computations at
%compile time, and has been shown to be Turing complete\cite{veldhuizen:templates_turing_complete}. 
%The key observation in the analogy to pattern matching is that partial and 
%explicit template specialization of \Cpp{} class templates are similar to 
%defining equations for Haskell functions. It is also used in \Cpp{} for similar 
%purposes: case analysis over families of types, structural decomposition of 
%types, overload resolution etc. It turns out we can use this compile-time 
%pattern matching facility to build a very expressive, efficient and extensible 
%run-time pattern matching solution.

%By efficient, we mean about as fast as functional languages for closed cases and
%much better than code generated for visitor patterns by commercial optimizers 
%for open cases\cite{TypeSwitch}.

\subsection{Summary}

We present functional-style pattern matching for \Cpp{} built as an ISO 
\Cpp{}11 library. Our solution:

\begin{compactitem}
\setlength{\itemsep}{0pt}
\setlength{\parskip}{0pt}
  \item Is open to introduction of new patterns into the library, while not 
        making any assumptions about existing ones.
  \item Is type safe: inappropriate applications of patterns to subjects are compile-time errors.
  \item Makes patterns first-class citizens in the language (\textsection\ref{sec:pat}).
  \item Is non-intrusive and can be retroactively applied to existing types (\textsection\ref{sec:bnd}).
\item Provides a unified syntax for various encodings of extensible 
       hierarchical datatypes in \Cpp{}.
  \item Provides an alternative interpretation of the controversial n+k 
        patterns (in line with that of constructor patterns), leaving the choice 
        of exact semantics to the user (\textsection\ref{sec:slv}).
  \item Supports a limited form of views (\textsection\ref{sec:view}).
  \item Generalizes open type switch to multiple scrutinees and enables patterns 
        in case clauses (\textsection\ref{sec:matchstmt}).
%  \item Is simpler than conventional object-oriented or union-based alternatives.
%  \item Improves performance compared to alternatives in real applications~\cite{TypeSwitch}.
  \item Demonstrates that compile-time composition of patterns through 
        concepts is superior to run-time composition of patterns through 
        polymorphic interfaces in terms of performance, expressiveness and 
        static type checking (\textsection\ref{sec:patcmp}).
\end{compactitem}

%The library in this respect is nothing else than a set of \Cpp{} concepts 
%guiding the composition of patterns combined with a set of macros providing a 
%suitable surface syntax. 

%We observe that the approach we use for open type switching generalizes easily 
%to the case of multiple scrutinees (subjects), while the closed approach does 
%not.

\noindent
Our library sets a standard for performance, extensibility, brevity, 
clarity and usefulness of any language solution for pattern matching.
It provides full functionality, so we can experiment with the use of 
pattern matching in \Cpp{} and compare it to existing alternatives.
Our solution requires only current support of \Cpp{}11 without any 
additional tool support.

%The rest of the paper is structured as follows. Section~\ref{sec:cpppat} 
%introduces common terminology and demonstrates the syntax enabled by the 
%\emph{Mach7} library~\cite{TS12}. Section~\ref{sec:impl} provides some 
%implementation details that make our extension possible. Section~\ref{sec:eval} 
%compares our solution to some alternatives. Section~\ref{sec:rw} discusses 
%related work, while Section~\ref{sec:cc} concludes by discussing some future 
%directions and possible improvements. Appendix~\ref{sec:prelim} provides some 
%details about \Cpp{} concepts and the syntax used in the paper.
