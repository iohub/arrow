\section{Conclusions and Future Work} %%%%%%%%%%%%%%%%%%%%%%%%%%%%%%%%%%%%%%%%%%
\label{sec:cc}

We present a pattern-matching library for C++ provides fairly standard
pattern-matching facilities. Our solution is 
non-intrusive and can be retroactively applied to any polymorphic or tagged 
class hierarchy. It also provides a uniform notation to these different 
encodings of algebraic and extensible hierarchical data types in C++.

We generalize n+k patterns to arbitrary expressions by letting the user define 
the exact semantics of such patterns. Our approach is more general than traditional approaches 
as it does not require an
equational view of such patterns. It also avoids hardcoding the 
exact semantics of n+k patterns into the language. 

We used the library to rewrite existing code that relied heavily on the 
visitor design pattern.
Our pattern matching code was much shorter (both source and object code), 
simpler, easier to maintain, comprehend, and faster. 
This confirmed our view of the visitor pattern as a clever workaround,
rather than a good solution to a fundamental problem.
The library approach was essential 
for experimentation in the context of real programs and for delivering 
performance comparable with or superior to conventional techniques in the 
context of industrial compilers.

The work presented here is only the beginning of our research on pattern 
matching for C++. We would like to experiment with other kinds of patterns, 
including those defined by the user; look at the interaction of patterns with 
other facilities in the language and the standard library; make
views less ad hoc etc. For example, standard containers in C++ do not have the 
implicit recursive structure present in data types of functional languages and 
viewing them as such with views would incur significant overheads. We will
experiment with very general patterns as first-class citizens.

Our generalization of n+k patterns depends on the properties of types involved 
in the expression. This should let us experiment not only with generic 
functions, but also with their generic inversions in the form of solvers. As 
more C++11 features become available in compilers it will also be interesting to 
look at how use of these features affects the ease of use, performance, 
readability, writability and debugging of the library and the user code that 
uses it.

In the nearest future, we would like to make our library to be safe and efficient 
in a multi-threaded environment. We would also like to look into providing an 
equally efficient and expressive support to multiple subjects, which in turn 
would allow us to address asymmetric multiple dispatch.
