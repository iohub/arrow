\section{Introduction} %%%%%%%%%%%%%%%%%%%%%%%%%%%%%%%%%%%%%%%%%%%%%%%%%%%%%%%%%
\label{sec:intro}

%Motivate the problem
%Give a summary of the paper: what you did and how
%Explicitly state your contribution

Classic algebraic data types as seen in functional languages are closed and their 
variants are disjoint, which allows for efficient implementation of case analysis 
on such types. Data types in object-oriented languages are extensible
and hierarchical (implying that variants are not necessarily 
disjoint). 
To be general, case analysis on such types must be \emph{open}: allow for 
independent extensions, modular type-checking, and dynamic linking. To be 
accepted for production code, its implementation must also equal or outperform 
all known workarounds. Existing approaches to case analysis on hierarchical 
extensible data types are either efficient or open, but not both.
Truly open approaches rely on expensive class-membership tests combined with 
decision trees. Efficient approaches rely on sealing either the class 
hierarchy or the set of functions, which loses extensibility (\textsection\ref{sec:prev}). %~\cite{Cohen91,DesignPatterns1993,Vitek97,PQEncoding}.  
Consider for example a simple expression language: 
\begin{displaymath}
\SynCat{exp} \is{} \SynCat{val} 
  \Alt{} \SynCat{exp} + \SynCat{exp}
  \Alt{} \SynCat{exp} - \SynCat{exp}
  \Alt{} \SynCat{exp} * \SynCat{exp}
  \Alt{} \SynCat{exp} / \SynCat{exp}
\end{displaymath}

\noindent 
In an object-oriented language without direct support for algebraic data types, 
the type representing an expression tree in the language will typically be 
encoded as an abstract base class, %listing the (sealed set of) allowed virtual functions, 
with derived classes implementing specific variants:
%
\begin{lstlisting}[keepspaces,columns=flexible]
struct Expr { virtual int eval() = 0; };
struct Value : Expr { @$\cdots$@ int eval(); int value; };
struct Plus  : Expr { @$\cdots$@ Expr& e1; Expr& e2; };
\end{lstlisting}

\noindent
A simple evaluator for this language can be implemented with the aid of a
virtual function \code{eval()} declared in the base class \code{Expr}. 
The approach is \emph{intrusive}, however, as the base class has to be
modified every time we add a new functionality. 
With \emph{Mach7}, we offer an external introspection of objects with case analysis:
%A less-intrusive solution can be achieved 
%with \emph{visitor design pattern}~\cite{DesignPatterns1993}, which tries to 
%leverage the case analysis common to all such functions. That solution, however, 
%is not open as it restricts extensibility of classes.

%Our solution overcomes the limitations of visitor design pattern by allowing 
%external introspection of objects with case analysis:

\begin{lstlisting}[columns=flexible]
int eval(const Expr& e)
{
  Match(e)
   Case(const  Value& x) return x.value;
   Case(const   Plus& x) return eval(x.e1)+eval(x.e2);
   Case(const  Minus& x) return eval(x.e1)-eval(x.e2);
   Case(const  Times& x) return eval(x.e1)*eval(x.e2);
   Case(const Divide& x) return eval(x.e1)/eval(x.e2);
  EndMatch
}
\end{lstlisting}

\noindent
This is all a user of \emph{Mach7} has to write.
The syntax is provided without any external tool support or 
additional definitions from the user. The library is implemented in
standard ISO \Cpp{}11~\cite{C++11} with template meta-programming, 
and a few macros (\textsection\ref{sec:vtblmem}). It runs 
about as fast as OCaml and Haskell equivalents (\textsection\ref{sec:ocaml}),
and occasionally
%, depending on the usage scenario, compiler and underlying hardware, 
comes close or 
outperforms the handcrafted \Cpp{} code based on the \emph{visitor design pattern} 
(\textsection\ref{sec:viscmp}).

The ideas and the 
\emph{Mach7} library were motivated by our unsatisfactory experiences working 
with various \Cpp{} front-ends and program analysis frameworks~\cite{Pivot09,gdr-2012:liz,Phoenix,Clang}. 
The problem was not in the frameworks per se, but in the fact that we had to use
the \emph{visitor design pattern}~\cite{DesignPatterns1993} to inspect, traverse, and 
elaborate abstract syntax trees of target languages. We found visitors 
unsuitable to express application logic directly, surprisingly hard to teach 
students, and often slower than handcrafted workaround techniques. Instead, 
users were relying on dynamic casts in many places, often nested, thus preferring 
shorter, cleaner, and more direct code to visitors. The consequential 
performance penalty was usually not discovered until later, when it was hard to 
remedy.

\subsection{Summary}

This paper makes the following contributions:
%
  \begin{itemize}
  \setlength{\itemsep}{0pt}
  \setlength{\parskip}{0pt}
  \item A technique for implementing open and efficient type 
        switching on extensible hierarchical data types as seen in 
        object-oriented languages.
  \item The technique delivers equivalent performance to that 
        of closed algebraic data types, and outperforms the visitor design pattern for open class hierarchies.
  \item A library implementation provides the notational convenience of
        functional-language constructs for selecting among types.
  \item A new approach to type switching that combines subtype tests and type 
        conversion, outperforming approaches that 
        combine subtype tests (even constant-time ones) with decision trees, even over
        small class hierarchies. 
  \item Complete integration with the existing \Cpp{} abstraction mechanisms and 
        object model.
  \item A constant-time function that partitions 
        a set of objects with the same static type into equivalence classes 
        based on the inheritance path of the static type within the dynamic type.
%  \item We demonstrate that combining (even constant-time) subtype tests with 
%        decision trees to the implementation of type switching is inferior in 
%        performance to workaround techniques like visitor design pattern and our 
%        approach for even small class hierarchies.
%  \item We advocate for taking the (space and time) complexity of type casts 
%        into account when studying techniques for subtype testing, type 
%        switching and run-time dispatch as these two operations are highly 
%        coupled in the presence of multiple inheritance.
  \end{itemize}

\noindent
In particular, our technique for efficient type switching:
%
  \begin{itemize}
  \setlength{\itemsep}{0pt}
  \setlength{\parskip}{0pt}
  \item Is open by construction (\textsection\ref{sec:poets}), non-intrusive, 
        and avoids the control inversion, as typical for visitors. 
  \item Works in the presence of multiple inheritance, both repeated and 
        virtual, as well as in generic code (\textsection\ref{sec:vtblmem}).
  \item Comes close to and often significantly outperforms the workaround techniques used in 
        practice, e.g. the visitor design pattern, without sacrificing extensibility
        (\textsection\ref{sec:viscmp}). 
  \item Does not require any changes to the \Cpp{} object model or computations at 
        link or load time. 
  \item Can be used in object-oriented languages with 
        object models similar to that of \Cpp{}.
  \item Is implemented as a library written in ISO Standard \Cpp{}11.
%  \item It applies to polymorphic (\textsection\ref{sec:vtp}-\ref{sec:vtblmem}) and 
%        tagged (\textsection\ref{sec:cotc}) class hierarchies through a unified  
%        syntax~\cite[\textsection 8]{TR}.
%  \item Our memoization device (\textsection\ref{sec:memdev}) generalizes to 
%        other languages and can be used to implement type switching 
%        (\textsection\ref{sec:vtblmem}), type testing 
%        (\textsection\ref{sec:poets},\cite[\textsection 9]{TR}), predicate dispatch 
%        (\textsection\ref{sec:memdev}), and multiple dispatch efficiently.
%  \item We list conditions under which virtual table pointers, commonly used in 
%        \Cpp{} implementations, uniquely identify the exact subobject within the 
%        dynamic type (\textsection\ref{sec:vtp}).
%  \item We also build an efficient cache indexing function for virtual table 
%        pointers that minimizes the number of conflicts 
%        (\textsection\ref{sec:sovtp},\ref{sec:moc}).
  \end{itemize}

\noindent
This is the first technique for handling type switching efficiently 
in the presence of the general multiple inheritance present in \Cpp{}.
Being a library, \emph{Mach7} can be used with any 
ISO \Cpp{}11 compiler (e.g., Microsoft, GNU, or Clang) without requiring 
any additional tools or preprocessors. It
sets a new threshold for acceptable performance, brevity, clarity, and
usefulness of a forthcoming compiler implementation of the open type switch in \Cpp{}.

%To the best of our knowledge this is the first technique capable of handling 
%type switching in the presence of multiple inheritance in its full generality 
%present in \Cpp{}. Prior work was mainly concentrating on constant-time type 
%inclusion tests under assumptions that were oversimplifying the reality of \Cpp{}.
% and omitted discussion of type casts typically following such 
%tests. We argue that in the presence of multiple inheritance the implementation 
%of casts may be significantly slower than that of subtype tests and thus the 
%complexity of the combined operation should be addressed by all the type 
%inclusion testing techniques. Besides, approaches based on constant-time subtype 
%tests require changes to object model and rely on load-time computations to 
%ensure openness. In contrast, our approach requires neither changes to the \Cpp{} 
%object model nor any computations at link or load time. We demonstrate that 
%applying even the fastest subtype tests to the implementation of type switching 
%is inferior in performance to visitor design pattern and our approach for even 
%small class hierarchies.
%
%A practical benefit of our library solution is that it can be used right away with any 
%compiler with a decent support of \Cpp{}11 without requiring the installation of 
%any additional tools or preprocessors. The solution is a proof of concept that 
%sets a minimum threshold for the performance, brevity, clarity and usefulness of 
%a language solution for open type-switching in \Cpp{}.

%Due to page restrictions, we will only deal here with the part of the library 
%concerned with efficient type testing, type identification, and type switching. 
%These primitive operations are essential in providing support for type patterns 
%and constructor patterns, and their efficiency relative to the visitor design 
%pattern might become a major argument for their adoption. We refer the reader to 
%technical report~\cite{TR} for more details on our pattern-matching 
%library.

%Section~\ref{sec:probl} will motivate the problem as well as discuss some 
%existing solutions and their limitations. Section~\ref{sec:copc} will outline 
%the three techniques that comprise our solution, while the 
%section~\ref{sec:eval} will evaluate their performance. Related work is 
%discussed in Section~\ref{sec:rw}, followed by concluding remarks in 
%Section~\ref{sec:cc}.
