\section{Implementation} %%%%%%%%%%%%%%%%%%%%%%%%%%%%%%%%%%%%%%%%%%%%%%%%%%%%%%%
\label{sec:impl}

We begin by explaining the traditional object-oriented approach to implementing 
first class patterns, which is based on run-time compositions through 
interfaces, and point to its problems and shortcomings. We then explain our 
approach based on compile-time composition through concepts. Finally, we detail 
how the patterns fit into and extend our open type switch statement~\cite{TS12}. 

\subsection{Patterns as Objects}
\label{sec:pao}

The idea of implementing patterns through objects is not new and has been 
explored before in different languages~\cite{Visser06matchingobjects,geller2010pattern,FuncCSharp,Grace2012}.
Variations of it differ in where bindings are stored and what is returned as a 
result of the match, but in its most basic form it consists of the following 
interface:

\begin{lstlisting}[keepspaces,columns=flexible]
struct object { // root of the object class hierarchy
  virtual bool operator==(const object&) const = 0;
};
@\halfline@
struct pattern { // pattern interface
  virtual bool match(const object&) = 0;
};
\end{lstlisting}

\noindent
A particular kind of pattern then implements \code{pattern} interface 
accordingly to its logic. A value pattern, for example, will look as following:

\begin{lstlisting}[keepspaces,columns=flexible]
struct value : pattern {
  value(const object& obj) : m_obj(obj) {}
  virtual bool match(const object& obj) 
    { return m_obj == obj; }
  const object& m_obj;
};
\end{lstlisting}

\noindent
The dual approach of passing a pattern to an object has also been explored, 
notably by Newspeak~\cite{geller2010pattern}, however its default implementation 
(which is rarely overridden in practice) dispatches back to the first solution. 
Regardless of whether the emphasis is made on the pattern or on the object, however,
the solution suffers from the same bottleneck -- the cost of a virtual function 
call(s) and the cost of identifying the dynamic type of either the subject or 
the pattern. We demonstrate in \textsection\ref{sec:patcmp}, the overhead 
incurred by this solution is incomparably larger than the overhead of our 
\emph{patterns as expression templates} approach.

While the approach is open to new patterns and pattern combinators (the patterns 
are composed at run-time by holding references to other pattern objects), it has 
some design problems. For example, mismatch in the type of the subject and the 
type accepted by the pattern can only be detected at run-time, while in 
languages with built-in support of pattern matching it is typically detected at 
type-checking phase. The approach may also unnecessarily clutter the code by 
requiring lots of similar boilerplate code be written. For example, modeling n+k 
patterns requires additional interface for evaluating the \code{expression}. 
With it, we have a dilema of whether \code{expression} should be derived from 
\code{pattern}, \code{pattern} from \code{expression}, or neither of those. 
Independently of the choice, implementation of pattern combinators will require 
that the class of the combinator conditionally derives from \code{pattern}, 
\code{expression} or both depending on which of these interfaces its arguments 
implement. On one hand, this will require a separate implementation of the 
combinator for each of the cases, while on the other it makes the combinators 
dependent on something that was only needed to implement n+k patterns.

%To quantify the overhead somewhat, we reimplemented the factorial function from 
%\textsection\ref{sec:cpppat} using object patterns and timed a million 
%computations of factorial on arguments ranging from 0 to 10. Depending on the 
%argument, the approach based on object patterns was 12-22 times slower than 
%factorial based on \emph{Mach7}. Note that for this experiment we took extra care to 
%not allocate patterns or intermediary objects on the heap, made sure the bodies 
%of all virtual functions were also available for inlining since we composed 
%objects on the stack and thus their complete types were known. We also used a 
%faster \code{typeid}-check instead of a slower \code{dynamic_cast} to ensure the 
%safety of unpacking an object. Finally, we repeated the experiment while removing 
%the safety check altogether (assuming the argument will be of the correct 
%dynamic type) and could reduce the overhead to 6.74-18 times, which is still too 
%costly to be considered a viable solution for a modern \Cpp{} use. We show in 
%\textsection\ref{sec:patcmp} that \emph{Mach7} patterns produce code that is only few 
%percentage points slower than manualy hand-crafted code without patterns. 

\subsection{Patterns as Expression Templates}
\label{sec:pat}

Patterns in \emph{Mach7} are also represented as objects, however their 
composition happens at compile time and is based on \Cpp{} concepts. 
\emph{Concept} is the \Cpp{} community's long-established term for a requirement 
for a set of template parameters. We define them briefly in Appendix~\ref{sec:prelim}, 
but otherwise follow the proposed notation for the forthcoming \Cpp{}14 
amendment~\cite{Concepts2012}. There are two main concepts on which the entire 
library is build: \code{Pattern} and \code{LazyExpression}.

\begin{lstlisting}
concept Pattern<typename P> = 
  Copyable<P> &&
  requires (typename S, P p, S s) {
    bool = { p(s) };
    AcceptedType<P,S>;
  };
\end{lstlisting}

\noindent
\code{Pattern} concept is the analog of \code{pattern} interface from the 
\emph{patterns as objects} solution. Any class modeling this concept can 
interact with other patterns of the library and \code{Match}-statement. Patterns 
can be passed as arguments of a function, which is why the concept refines a
\code{Copyable} concept. Patterns require presence of an application operator 
that serves as an analog of \code{pattern::match(const object&)} interface 
method in the \emph{patterns as objects} approach; there is no analog of the 
\code{object} interface, however. The \code{Pattern} concept itself does not 
impose any restrictions on the type of the subject \code{S}. Patterns like 
wildcard pattern will leave this type completely unrestricted, while other 
patterns may require it to model a certain concept, inherit from a certain type, etc. 
Application operator will typically return a value of type \code{bool} 
indicating whether the pattern is \emph{accepted} on a given subject 
(\code{true}) or \emph{rejected} (\code{false}). For convenience reasons, 
application operator is allowed to return any type that is convertible to 
\code{bool} instead, e.g. a pointer to a casted subject, which is useful in 
emulating the support of \emph{as-patterns}.

Most of the patterns are applicable only to subjects of a given \emph{expected 
type} or types convertible to it. This is the case, for example, with value and 
variable patterns, where the expected type is the type of the underlying value, 
as well as with constructor pattern, where the expected type is the type of the 
user-defined type it decomposes. Some patterns, however, do not have a single 
expected type and may work with subjects of many unrelated types. A wildcard 
pattern, for example, can accept values of any type without involving a 
conversion. To account for this, the \code{Pattern} concept requires presence of 
a type alias \code{AcceptedType}, which given a pattern of type \code{P} and 
a subject of type \code{S} returns an expected type \code{AcceptedType<P,S>} 
that will accept subjects of type \code{S} with no or minimum conversions. 
By default, the alias is defined in terms of a nested type-function 
\code{accepted_type_for} as following:

\begin{lstlisting}
template<typename P, typename S>
  using AcceptedType = P::accepted_type_for<S>::type;
\end{lstlisting}

\noindent
The wildcard pattern defines \code{accepted_type_for} to be an identity 
function, while variable and value patterns define it to be their underlying 
type. Here is an example of how variable pattern models the concept: 

%struct wildcard {
%  template <typename S>
%  struct accepted_type_for { typedef S type; };
%  template <typename S> 
%  bool operator()(const S&) const noexcept 
%    { return true; }
%};
%@\halfline@
%template <typename T>
%struct value {
%  template <typename S> 
%  struct accepted_type_for { typedef T type; };
%  bool operator()(const T& t) const noexcept 
%    { return m_value == t; }
%  T m_value;
%};
%@\halfline@
\begin{lstlisting}
template <Regular T>
struct var {
  template <typename S> 
    struct accepted_type_for { typedef T type; };
  bool operator()(const T& t) const 
    { m_value = t; return true; }
  template <Regular S> 
  bool operator()(const S& s) const
    { m_value = s; return m_value == s; }
  mutable T m_value;
};
\end{lstlisting}

%Each of our six pattern kinds implements the application operator according to 
%the semantics presented in Figure~\ref{exprsem}. The application operator's 
%result has to be convertible to bool; \code{true} indicates a successful match. 
%A class might have several overloads of the above operator that distinguish 
%cases of interest. We summarize the requirements on template parameters of each 
%of our pattern in Figure~\ref{xt-reqs}.
%
%\begin{figure}[h]
%\centering
%\begin{tabular}{llll}
%{\bf Pattern}       & {\bf Parameters}          & {\bf Argument of application operator U}         \\ \hline
%\code{wildcard}     & --                        & --                                               \\
%\code{value<T>}     & \code{Regular<T>}         & \code{Convertible<U,T>}                          \\
%\code{variable<T>}  & \code{Regular<T>}         & \code{Convertible<U,T>}                          \\
%\code{expr<F,E...>} & \code{LazyExpression<E>}  & \code{Convertible<U,expr<F,E...>::result_type>}  \\
%\code{guard<E1,E2>} & \code{LazyExpression<Ei>} & any type accepted by \code{E1::operator()}       \\
%\code{ctor<T,E...>} & \code{Polymorphic<T>}     & \code{Polymorphic<U>} for open encoding          \\
%                    & \code{Object<T>}          & \code{is_base_and_derived<U,T>} for tag encoding \\
%\end{tabular}
%\caption{Requirements on parameters and argument type of an application operator}
%\label{xt-reqs}
%\end{figure}

\noindent
Note that for semantic or efficiency reasons a pattern may have several overloads 
of application operator as in the example above. The first one is used when no 
conversion is required and thus the pattern is guaranteed to be accepted, while 
the second may involve a possibly narrowing conversion, which is why we check 
that the values compare equal after assignment. Similarly, for type checking 
reasons, \code{accepted_type_for} may and typically will provide several partial 
or full specializations to limit the set of acceptable subjects. For example, 
address combinator can only be applied to subjects of pointer types. Its 
implementation manifests this by deriving unrestricted case of the type function 
\code{accepted_type_for} from \code{invalid_subject_type<S>}. This will trigger 
a static assertion when its associated type \code{type} gets instantiated, 
resulting in a compile-time error that states that a given subject type \code{S} 
cannot be used as an argument of the address pattern. The second case of the 
type function indicates through partial specialization of class templates that 
for any subject of a pointer type \code{S*}, the accepted type is going to be a 
pointer to the type accepted by the argument pattern \code{P1} of the address 
combinator.

\begin{lstlisting}
template <Pattern P1>
struct address
{ // ...
  template <typename S> 
    struct accepted_type_for : invalid_subject_type<S> {};
  template <typename S> struct accepted_type_for<S*> {
    typedef typename P1::template 
      accepted_type_for<S>::type* type;
  };
  template <typename S>
    bool operator()(const S* s) const 
      { return s && m_p1(*s); }
  P1 m_p1;
};
\end{lstlisting}

\noindent
A slight inconvenience may arise from the fact that pattern's implementation is 
expected to provide a set of overloads of the application operator that will be 
able to accept all the valid subject types, however the benefits of specifying 
type checking rules locally to the pattern significantly outweights this setback.

Guard and n+k patterns, as well as equivalence combinator and potentially some 
new user-defined patterns, depend on capturing the structure (term) of lazily 
evaluated expressions. All such expressions model \code{LazyExpression} concept: 

\begin{lstlisting}
concept LazyExpression<typename E> = 
  Copyable<E> &&
  requires (E e) {
    ResultType<E>;
    ResultType<E> == { eval(e) };
    ResultType<E> { e };
  };
\end{lstlisting}

\noindent
Associated type \code{result_type} defines the type of the result of a lazily 
evaluated expression. Any model of \code{LazyExpression} must also provide an 
implementation of function \code{eval} that evaluates the result of the 
expression. The concept also provides default implementation of the conversion 
to \code{result_type} in order to allow the use of lazily evaluated expressions 
in the contexts where their eagerly computed value is expected: e.g. non-pattern 
matching context of the right hand side of the \code{Case} clause. Class 
\code{var<T>}, for example, models concept \code{LazyExpression} as following:

\begin{lstlisting}
template <typename T>
struct var {
  // ... definitions from before
  typedef T result_type;
  friend const result_type& eval(const var& v) 
    { return v.m_value; }
  friend       result_type& eval(      var& v) 
    { return v.m_value; }
};
\end{lstlisting}

\noindent
To capture the structure of an expression, the library employs a commonly used 
technique called ``expression templates''~\cite{Veldhuizen95expressiontemplates, 
vandevoorde2003c++}. It captures the structure of expression through the type, 
which for binary addition may look as following:

\begin{lstlisting}[keepspaces,columns=flexible]
template <LazyExpression E1, LazyExpression E2>
struct plus {
  E1 m_e1; E2 m_e2; // subexpressions
  plus(const E1& e1, const E2& e2) : m_e1(e1), m_e2(e2) {}
  typedef decltype(std::declval<E1::result_type>() 
                 + std::declval<E2::result_type>()
                  ) result_type;
  friend result_type eval(const plus& e) 
    { return eval(e.m_e1) + eval(e.m_e2); }
  friend plus operator+(const E1& e1, const E2& e2)
    { return plus(e1,e2); }
};
\end{lstlisting}

\noindent
The user of the library never sees this definition, instead she implicitly 
creates its objects with the help of overloaded \code{operator+} on any 
\code{LazyExpression} arguments. The type itself models \code{LazyExpression} 
concept as well so that the lazy expressions can be composed. Notice that all 
the requirements of the concept are implemented in terms of the requirements 
on the types of the arguments. The key point to efficiency of expression 
templates is that all the types in the final expression are known at compile 
time, while all the function calls are trivial and fully inlined. Use of new 
\Cpp{}11 features like move constructors and perfect forwarding allows us to 
ensure further that no temporary objects will ever be created at run-time and 
that the evaluation of the expression template will be as efficient as a hand 
coded function.

In general, an \emph{expression template} is an algebra \\ $\langle T_C,\{f_1,f_2,...\}\rangle$ 
defined over the set $T_C = \{\tau | \tau \models C\}$ of all the types $\tau$ 
modeling a given concept $C$. Operations $f_i$ allow one to compose new types  
modeling concept $C$ out of existing ones. In this sense, the types of all lazy 
expressions in \emph{Mach7} stem from a set of few possibly parameterized basic 
types like \code{var<T>} and \code{value<T>} (which model \code{LazyExpression}) 
by applying type functors \code{plus}, \code{minus} ... etc. to them. Every type 
in the resulting family then has a function \code{eval} defined on it that 
returns a value of the associated type \code{result_type}. Similarly, the types 
of all the patterns stem from a set of few possibly parameterized patterns like 
\code{wildcard}, \code{var<T>}, \code{value<T>}, \code{C<T>} etc. by applying to 
them pattern combinators like \code{conjunction}, \code{disjunction}, 
\code{equivalence}, \code{address} etc. The user is allowed to extend both 
algebras with either basic expressions and patterns or functors and combinators. 

Sets $T_{LazyExpression}$ and $T_{Pattern}$ have non-empty intersection, which 
slightly complicates the matter. Basic types \code{var<T>} and \code{value<T>} 
belong to both families and so are some of the combinators: e.g. 
\code{conjunction}. Since we can only have one overloaded \code{operator&&} for 
a given combination of argument types, we have to model conditionally
\code{Pattern}, \code{LazyExpression} or both in a given instantiation of 
\code{conjunction<T1,T2>} depending on what combination of these concepts the 
argument types \code{T1} and \code{T2} model. Concepts, unlike interfaces, allow 
modeling such situations without multiplying implementations or introducing 
dependencies.

\subsection{Structural Decomposition}
\label{sec:bnd}

\emph{Mach7}'s support of constructor patterns \code{C<T>(P1,...,Pn)} requires the 
library to know which member of class \code{T} should be used as the subject to 
$P_1$, which should be matched against $P_2$ etc. In functional languages 
supporting algebraic data types, such decomposition is unambiguous as each 
variant has only one constructor, which is thus also used as \emph{deconstructor}
-- a term used by several languages~\cite{padl08,Thorn2012} to define explicitly
decomposition of a type through pattern matching. In \Cpp{}, a class may have 
several constructors, which may not even reflect the data members it contains, 
some of which may be inaccessible or unrepresentative of the class' logic etc. 
We thus require the user to be more explicit as to class' decomposition, which 
she does by specializing the library template class \code{bindings}. We decided 
not to use the name ``deconstructor'' as it is already commonly confused with 
``destructor'' in \Cpp{}. Here are the definitions required to decompose the 
lambda terms we introduced in \textsection\ref{sec:cpppat}:

\begin{lstlisting}
template <> struct bindings<Var> { Members(Var::name); };
template <> struct bindings<Abs> { Members(Abs::var, Abs::body); };
template <> struct bindings<App> { Members(App::func, App::arg); };
\end{lstlisting}

\noindent
Variadic macro \code{Members} simply expands each of its argument into the 
following definition, demonstrated here on \code{App::func}:

\begin{lstlisting}
static inline decltype(&App::func) member@$i$@() noexcept 
  { return &App::func; }
\end{lstlisting}

\noindent
Each of such functions returns a pointer-to-member that should be bound in 
position $i$. The library applies corresponding members to the subject in order 
to obtain subjects for sub-patterns $P_1,...,P_n$. The functions get inlined so 
the code to access a member in a given position becomes exactly the same as the 
code to access that member directly. Note that binding definitions made this way 
are \emph{non-intrusive} since the original class definition is not touched. 
They also respect \emph{encapsulation} since only the public members of the 
target type will be accessible from within \code{bindings} specialization. 
Members do not have to be data members only, which can be inaccessible, but any 
of the following:

\begin{compactitem}
\setlength{\itemsep}{0pt}
\setlength{\parskip}{0pt}
\item Data member of the target type $T$
\item Nullary member-function of the target type $T$
\item Unary external function taking the target type $T$ by pointer, reference or value.
\end{compactitem}

\noindent
Binding definitions have to be written only once for a given class hierarchy and 
can be used everywhere. This is also true for parameterized classes, as can be 
seen in an example in \textsection\ref{sec:view}.


\subsection{Algebraic Decomposition}
\label{sec:slv}

Traditional approaches to generalizing n+k patterns treat matching a pattern 
$f(x,y)$ against a value $v$ as solving an equation $f(x,y)=v$~\cite{OosterhofThesis}. 
Such an interpretation is well defined when there are zero or one solutions,
but alternative interpretations are possible when there are multiple solutions. 
Instead of discussing which interpretation is the most general or appropriate, 
we propose to look at n+k patterns as a \emph{notational decomposition} of 
mathematical objects. The elements of such notation are associated with 
sub-components of the matched mathematical entity, which effectively lets us 
decompose it into parts. The structure of the expression tree used in the notion
is an analog of a constructor symbol in structural decomposition, while its 
leaves are placeholders for parameters to be matched against or inferred from 
the mathematical object in question. In essence, \emph{algebraic decomposition} 
is to mathematical objects what structural decomposition is to algebraic data 
types. While the analogy is somewhat ad-hoc, it resembles the situation with 
operator overloading: you do not strictly need it, but it is so syntactically 
convenient it is virtually impossible not to have it. We demonstrate the 
alternative point of view on the n+k patterns with examples.

\begin{compactitem}
\setlength{\itemsep}{0pt}
\setlength{\parskip}{0pt}
\item An expression $n/m$ is often used to decompose a rational number into 
      numerator and denominator.
\item Euler notation $a+bi$ with $i$ being an imaginary unit is used to 
      decompose a complex number into real and imaginary parts. Similarly, 
      expressions $r(cos \phi + i\mathrm{sin} \phi)$ and $re^{i\phi}$ are used to 
      decompose it into polar form.
\item An object representing 2D line can be decomposed with slope-intercept form 
      $mX+c$, linear equation form $aX+bY=c$ or two-points form 
      $(Y-y_0)(x_1-x_0)=(y_1-y_0)(X-x_0)$.
\item An object representing polynomial can be decomposed for a specific degree: 
      $a_0$, $a_1X^1+a_0$, $a_2X^2+a_1X^1+a_0$ etc.
\item An element of a vector space can be decomposed along some sub-spaces of 
      interest. For example a 2D vector can be matched against $(0,0)$, $aX$, 
      $bY$, $aX+bY$ to separate the general case from those when one or both
      components of vector are $0$.
\end{compactitem}

\noindent
Expressions $i$, $X$ and $Y$ in these examples are not variables, but named 
constants of some dedicated type that lets the expression be generically 
decomposed into orthogonal parts. Note also that linear equation and two-point 
form for decomposing lines already include an equality sign, which makes it 
hard to give them semantics in an equational approach. It turns out that the 
equational approach can be generically expressed in our framework for many 
interesting cases of interest.

Applying equational approach to floating-point arithmetic creates even more 
problems. Even when the solution is unique, it may not be representable by 
a given floating-point type and thus not satisfy the equation. Once we settle 
for an approximation, we open ourselves to even more decompositions that become 
possible with our approach.

\begin{compactitem}
\setlength{\itemsep}{0pt}
\setlength{\parskip}{0pt}
\item Matching $n/m$ with integer variables $n$ and $m$ against a floating-point 
      value can be given semantics of finding the closest fraction to the 
      value.
\item Matching an object representing sampling of some random variable against
      expressions like $Gaussian(\mu,\sigma^2)$, $Poisson(\lambda)$ or 
      $Binomial(n,p)$ can be seen as distribution fitting. 
\item Any curve fitting in this sense becomes an application of pattern 
      matching. Precision in this case can be a global constant or explicitly 
      passed parameter of the matching expression.
\end{compactitem}

%\noindent
%We can make several observations from these examples:

%\begin{compactitem}
%\setlength{\itemsep}{0pt}
%\setlength{\parskip}{0pt}
%\item We might need to have the entire expression available to us in order to 
%      decompose its parts.
%\item Matching the same expression can have different meanings depending on 
%      types of objects composing the expression and the expected result. 
%\item An algorithm to decompose a given expression may depend on the types of 
%      objects in it and the type of the result. 
%\end{compactitem}

%\subsubsection{Solvers}

\noindent
The user of our library defines the semantics of decomposing a value of a given 
type \code{S} against an expression of shape \code{E} by overloading a function: 

\begin{lstlisting}
template <LazyExpression E, typename S> 
bool solve(const E&, const S&);
\end{lstlisting}

\noindent
The first argument of the function takes an expression template representing a 
term we are matching against, while the second argument represents the expected 
result. The following example defines a generic solver for multiplication by a 
constant:

\begin{lstlisting}
template <LazyExpression E, typename T> 
    requires Field<E::result_type>
bool solve(const mult<E,value<T>>& e, const E::result_type& r)
    { return solve(e.m_e1,r/eval(e.m_e2)); }
@\halfline@
template <LazyExpression E, typename T>
    requires Integral<E::result_type>
bool solve(const mult<E,value<T>>& e, const E::result_type& r) 
{
    T t = eval(e.m_e2);
    return r%t == 0 && solve(e.m_e1,r/t);
}
\end{lstlisting}

\noindent
The first overload is only applicable when the type of the result of the 
sub-expression models the \code{Field} concept. In this case, we can rely on the 
presence of a unique inverse and simply call division without any additional 
checks. The second overload uses integer division, which does not guarantee the 
unique inverse, and thus we have to verify that the result is divisible by the 
constant first. This last overload combined with a similar solver for addition 
of integral types is everything the library needs to know to properly handle the 
definition of the \code{fib} function from \textsection\ref{sec:cpppat}. It also 
demonstrates how an equational approach can be generically implemented for a 
number of expressions.

A generic solver capable of decomposing a complex value using the Euler 
notation is very easy to define by fixing the structure of expression:

\begin{lstlisting}[keepspaces]
template <LazyExpression E1, LazyExpression E2> 
    requires SameType<E1::result_type,E2::result_type>
bool solve(
    const plus<mult<E1,value<complex<E1::result_type>>>,E2>& e, 
    const complex<E1::result_type>& r);
\end{lstlisting}

\noindent
As we mentioned in \textsection\ref{sec:cpppat}, the template facilities of 
\Cpp{} resemble pattern-matching facilities of other languages. Here, we 
essentially use these compile-time patterns to describe the structure of the 
expression this solver is applicable to: $e_1*c+e_2$ with types of $e_1$ and 
$e_2$ being the same as type on which a complex value $c$ is defined. The actual 
value of the complex constant $c$ will not be known until run-time, but assuming 
its imaginary part is not $0$, we will be able to generically obtain the values 
for sub-expressions.

Our approach is largely possible due to the fact that the library only serves as 
an interface between expressions and functions defining their semantics and 
algebraic decomposition. The fact that the user explicitly defines the variables 
she would like to use in patterns is also a key as it lets us specialize not 
only on the structure of the expression, but also on the types involved. 
Inference of such types in functional languages would be hard or impossible as the 
expression may have entirely different semantics depending on the types of 
arguments involved. Concept-based overloading simplifies significantly the case 
analysis on the properties of types, making the solvers generic and composable.
The approach is also viable as expressions are decomposed at compile-time and 
not at run-time, letting the compiler inline the entire composition of solvers. 

An obvious disadvantage of this approach is that the more complex expression 
becomes, the more overloads the user will have to provide to cover all 
expressions of interest. The set of overloads will also have to be made 
unambiguous for any given expression, which may be challenging for novices. An 
important restriction of this approach is its inability to detect multiple uses 
of the same variable in an expression at compile time. This happens because 
expression templates remember the form of an expression in a type, so use of two 
variables of the same type is indistinguishable from the use of the same 
variable twice. This can be worked around by giving different variables 
(slightly) different types or making additional checks as to the structure of 
expression at run-time, but that will make the library even more verbose or 
incur a significant run-time overhead.

\subsection{Views}
\label{sec:view}

Any type $T$ may have an arbitrary number of \emph{bindings} associated with it, 
which are specified by varying the second parameter of the \code{bindings} 
template -- \emph{layout}. The layout is a non-type template parameter of an 
integral type that has a default value and is thus omitted most of the time.
Support of multiple bindings through layouts in our library effectively enables 
a facility similar to Wadler's \emph{views}\cite{Wadler87}. Consider:

\begin{lstlisting}
enum { cartesian = default_layout, polar }; // Layouts
@\halfline@
template <typename T> 
  struct bindings<std::complex<T>>
    { Members(std::real<T>,std::imag<T>); };
template <typename T> 
  struct bindings<std::complex<T>, polar>
    { Members(std::abs<T>,std::arg<T>); };
@\halfline@
template <typename T> 
  using Cartesian = view<std::complex<T>>;
template <typename T> 
  using Polar     = view<std::complex<T>, polar>;
@\halfline@
  std::complex<double> c; double a,b,r,f;
  Match(c)
    Case(Cartesian<double>>(a,b)) ... // default layout
    Case(    Polar<double>>(r,f)) ... // view for polar layout
  EndMatch
\end{lstlisting}

\noindent
The \Cpp{} standard effectively enforces the standard library to use cartesian 
representation\cite[\textsection26.4-4]{C++11}, which is why we choose the 
\code{Cartesian} layout to be the default. We then define bindings for each 
layout and introduce template aliases (an analog of typedefs for parameterized 
classes) for each of the views. \emph{Mach7} class \code{view<T,l>} binds together a 
target type with one of its layouts, which can be used everywhere where an 
original target type was expected.

The important difference from Wadler's solution is that our views can only be 
used in a match expression and not as a constructor or arguments of a function 
etc.

\subsection{Match Statement}
\label{sec:matchstmt}

\code{Match}-statement presented in this paper extends the efficient type switch 
for \Cpp{}~\cite{TS12}. The core of this extension amounts to ability to handle 
multiple subjects (both polymorphic and non-polymorphic) 
(\textsection\ref{sec:multiarg}) as well as the ability to accept patterns in 
case clauses (\textsection\ref{sec:patcases}).

\subsubsection{Multi-argument Type Switching}
\label{sec:multiarg}

The core of the proposal for efficient type switching was based on the fact that 
virtual table pointers (vtbl-pointers) on one hand uniquely identify subobjects 
in the object, while on the other hand they are perfect for hashing. The optimal 
hash function $H_{kl}^V$ built for a set of virtual table pointers $V$ seen by a 
type switch was chosen by varying parameters $k$ and $l$ to minimize the 
probability of conflict. Parameter $k$ was representing the logarithm of the 
size of cache, while parameter $l$ was representing the number of smallest bits 
to ignore.

We considered two different approaches to extending that solution to $N$ 
arguments. The first approach was based on maintaining an $N$-dimensional 
table indexed by independent $H_{k_il_i}^{V_i}$ maintained for each of the 
arguments $i$. The second approach was to aggregate the information from 
multiple vtbl-pointers into a single hash in a hope the hashing would still 
maintain its favorable properties. The first approach requires amount of memory 
proportional to $O(|V|^N)$ regardless of how many different combinations of 
vtbl-pointers came through the statement. The second approach requires the 
amount of memory linear in the number of vtbl-pointer combinations seen, which 
in the worst case becomes the same $O(|V|^N)$. The first approach requires 
lookup in $N$ caches, with each lookup being a subject to potential collisions; 
the second approach requires non-trivial computations to aggregate $N$ 
vtbl-pointers into a single hash value and may result in potentially more 
collisions in comparison to the first approach. Our experience of dealing with 
multiple dispatch in \Cpp{} suggests that we rarely see all combinations of 
types coming through a given multi-method in real-world applications. With this 
in mind, we did not expect all combination of types come through a given 
\code{Match}-statement and thus preferred the second solution, which grows 
linearly in memory with the number of combinations seen.

In number theory, \emph{Morton order} (aka \emph{Z-order}) is a function that 
maps multidimensional data to one dimension while preserving locality of the 
data points~\cite{Morton66}. A Morton number of an $N$-dimensional coordinate 
point is obtained by interleaving the binary representations of all coordinates.
The original one-dimensional hash function $H_{kl}^V$ applied to arguments $v \in V$ 
was producing hash values in a tight range $[0..2^k[$ where $k \in [K,K+1]$ for 
$2^{K-1} < |V| \leq 2^K$. The produced values were close to each other, which 
helped improve the performance of cache due to locality of references. The 
idea is thus to use Morton order on these hash values and not on the original 
vtbl-pointers in order to maintain the locality of references. To do this, we 
still maintain a single parameter $k$ reflecting the size of cache, however we 
keep $N$ parameters $l_i$ -- an optimal offset for argument $i$.

Consider a set $V^N = \{\tpl{v_1^1,...,v_1^N},...,\tpl{v_n^1,...,v_n^N}\}$ of 
$N$-dimensional tuples representing the set of vtbl-pointer combinations coming 
through a given \code{Match}-statement. As with one-dimensional case, we 
restrict the size $2^k$ of the cache to be not larger than twice the closest 
power of two greater or equal to $n=|V^N|$: i.e. $k \in [K,K+1]$, where 
$2^{K-1} < |V^N| \leq 2^K$. For a given $k$ and offsets $l_1,...,l_N$ a hash 
value of a given combination $\tpl{v^1,...,v^N}$ is defined as 
$H_{kl_1...l_N}(\tpl{v^1,...,v^N})=\mu(\frac{v^1}{2^{l_1}},...,\frac{v^N}{2^{l_N}}) \mod 2^k$, 
where function $\mu$ returns Morton number (bit interleaving) of $N$ numbers.
 
Similar to one-dimensional case, we vary parameters $k,l_1,...,l_N$ in 
their finite and small domains to obtain an optimal hash function 
$H^{V^N}_{kl_1...l_N}$ by minimizing the probability of conflict on values from 
$V^N$. Unlike the one-dimensional case, we do not try to find the optimal 
parameters every time we reconfigure the cache. Instead, we only try to improve 
the parameters to render fewer conflicts in comparison to the number of conflicts 
rendered by the current configuration. This does not prevent us from eventually 
converging to the same optimal parameters, which we do over time, but is 
important for maintaining the amortized complexity of the access constant. 
%Observe that the domain of each parameter of the optimal hash function 
%$H^{V^N}_{kl_1...l_N}$ only grows since $V^N$ only grows, while any cache 
%configuration is also a valid cache configuration in a larger cache, rendering 
%the same number of conflicts.
We demonstrate in \textsection\ref{sec:morton} that similarly to one-dimensional 
case such hash function produces little collisions on real-world class 
hierarchies, while is simple enough to compute to compete with alternatives 
dealing with multiple dispatch.

%In practice, the library does not consider all $N$ arguments of a given 
%\code{Match}-statement, but only the $M$ polymorphic arguments ($M \leq N$). It 
%then builds an efficient type switch based on those $M$ arguments. The type 
%switch guarantees efficient dispatch to the first case clause that can possibly 
%handle a given combination of arguments based on the subset of only polymorphic 
%arguments. The patterns are then tried sequentially. The underlying type switch 
%uses pattern's type-function \code{accepted_type_for<Si>} instantiated with the 
%subject type $Si$ of a given argument $i$ in order to obtain the target type 
%requested by the pattern in that position.

\subsubsection{Support for Patterns}
\label{sec:patcases}

Given a statement \code{Match(x_1,...,x_N)}, the library introduces several 
names into the scope of the statement: e.g. number of arguments $N$, subject 
types \code{subject_type_i} (defined as \code{decltype(x_i)} modulo type 
qualifiers), number of polymorphic arguments $M$ etc. When $M > 0$ it also 
introduces the necessary data structures to implement efficient type 
switching~\cite{TS12}. Only the $M$ arguments whose \code{subject_type_i} are 
polymorphic will be used for fast type switching.

For each case clause \code{Case(p_1,...,p_N)} the library ensures that the 
number of arguments to the case clause $N$ matches the number of arguments to 
the \code{Match} statement, and that the type \code{P_i} of every expression 
\code{p_i} passed as its argument models the \code{Pattern} concept. 
Initially we allowed case clauses to accept less than $N$ patterns, assuming the 
missing patterns to be the wildcard, however, brittleness of the macro system 
made us reconsider this. The problem is that macro system is blind to \Cpp{} 
syntax and template instantiation like \code{A<B,C>} used in a pattern will be 
treated by the preprocessor as 2 macro arguments. This resulted in errors that 
were hard for the users to comprehend.
For each \code{subject_type_i} it then introduces \code{target_type_i} into the 
scope of the case clause, defined as the result of evaluating type function 
\code{P_i::accepted_type_for<subject_type_i>}. This is the type the pattern 
expects as an argument on the subject of type \code{subject_type_i} (\textsection\ref{sec:pat}), 
which is used by the type switching mechanism to properly cast the subject if necessary. 
The library then introduces names \code{match_i} of type \code{target_type_i&} 
bound to properly casted subjects and available to the user in the right-hand 
side of the case clause in case of a successful match. The qualifiers applied to 
the type of \code{match_i} reflect the qualifiers applied to the type of subject 
\code{x_i}. Finally, the library generates the code that sequentially checks 
each pattern on properly casted subjects, making the clause's body conditional:

\begin{lstlisting}
if (p_1(match_1) && ... && p_N(match_N)) { /* body */ }
\end{lstlisting}

\noindent
When type switching is not involved, the generated code implements the naive 
backtracking strategy, which is known to be inefficient as it can produce 
redundant computations~\cite[\textsection 5]{Cardelli84}. More efficient 
algorithms for compiling pattern matching have been developed 
since~\cite{Augustsson85,Maranget92,Puel93,OPM01,Maranget08}. While these 
algorithms cover most of the typical kinds of patterns, they are not pattern agnostic 
as they make assumptions about semantics of concrete patterns. A library-based 
approach to pattern matching is agnostic of the semantics of any given 
user-defined pattern. The interesting research question in this context would 
be: what language support is required to be able to optimize first class 
patterns. While we do not address this question in its generality, our solution 
makes a small step in that direction.

The main advantage from using pattern matching in \emph{Mach7} comes from the fast type 
switching weaved into the \code{Match}-statement. It effectively skips case 
clauses that will definitely be rejected because their target types are not 
subtypes of subjects' dynamic types. This, of course, is only applicable to 
polymorphic arguments, for non-polymorphic arguments the matching is done 
naively with cascade of conditional statements.
